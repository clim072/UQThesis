% ***************************************************
% How to introduce a previously published chapter
% ***************************************************
%This is an example of how you might introduce a chapter that has been published previously. 
\cleartoevenpage
%\pagestyle{empty}	
%Use this command (above) to suppress the header from the preceding chapter.

\noindent
The following publication has been incorporated as Chapter~\ref{Chap:label}.

\noindent
1.~\cite{DumyCitationKey} \textbf{Your Name}, Co-author 1, and Final Author, \href{linktoyourpaper}{Title of your paper}, \textit{Journal} Issue, Number, Year

\begin{table}[h]
	\begin{center}
	\begin{tabular}{|c|l|l|}
		\hline
		Contributor & Statement of contribution & \% \\
		\hline
		\textbf{Your Name}				& writing of text 					& 70\\
															& proof-reading							& 60 \\
															& theoretical derivations 	& 70\\
															& numerical calculations 		& 100\\
															& preparation of figures 		& 80 \\
															& initial concept						& 10 \\
		\hline
		Co-author 1								& writing of text 					& 20\\
															& proof-reading							& 10 \\
															& supervision, guidance 		& 20\\
															& theoretical derivations 	& 10\\
															& preparation of figures 		& 20 \\
															& initial concept						& 10 \\
		\hline
		Final Author							& writing of text 					& 10\\
															& proof-reading							& 30 \\
															& supervision, guidance 		& 80 \\
															& theoretical derivations 	& 20 \\
															& preparation of figures 		& 10 \\
															& initial concept						& 80 \\
		\hline
	\end{tabular}
	\end{center}
\end{table}

If your task breakdown requires further clarification, do so here. Do not exceed a single page.


% ***************************************************
% Example of an internal chapter
% ***************************************************
%This is an internal chapter of the thesis.
%If you have a long title, you can supply an abbreviated version to print in the Table of Contents using the optional argument to the \chapter command.
\chapter[Abbreviated title]{Full title}
\label{Chap:label}	%CREATE YOUR OWN LABEL.
\pagestyle{headings}

%If you are presenting work which has been previously published, acknowledge this here.

% ********* Enter your text below this line: ********
Introduce the broad layout of the chapter.

% ***************************************************


\section{Introduction}
\label{Sec:label}	%CREATE YOUR OWN LABEL.

% ********* Enter your text below this line: ********
Add your text here. 

% ***************************************************
